\documentclass[12pt]{amsart}
\setlength{\parskip}{.1in}
\setlength{\parindent}{0cm}
%myalterations
\usepackage{amssymb}
\usepackage{amsmath}
\usepackage[usenames,dvipsnames,svgnames,table]{xcolor}
\usepackage[colorlinks=true,urlcolor=blue,pdfborder={0 0 .5}pdfnewwindow=true]{hyperref}
\usepackage{enumitem}
%\usepackage{amsthm}
\usepackage{graphicx}
\usepackage{verbatim}
\usepackage{tabularx}
%\usepackage{arydshln,leftidx,mathtools}
\usepackage{bm}
\usepackage{tikz}
\usepackage{tikz-cd}
\usepackage{hyperref}
\usepackage{bm}
\usepackage{mathtools}
\usepackage{stmaryrd}

%\setlength{\dashlinedash}{.4pt}
%\setlength{\dashlinegap}{.8pt}
%\usepackage{amsthm}
\usepackage{verbatim}
%\usepackage{commath}
%My commands
%environment abbreviations
\newcommand{\benu}{\begin{enumerate}}
\newcommand{\eenu}{\end{enumerate}}
\newcommand{\bed}{\begin{description}}
\newcommand{\ed}{\end{description}}
\theoremstyle{definition}
\newtheorem{theorem}{Theorem}
\newtheorem{notation}[theorem]{Notation}
\newtheorem{exercise}[theorem]{Exercise}
\newcommand{\bex}{\begin{exercise}}
\newcommand{\ex}{\end{exercise}}

\newcommand{\pru}{{ \bfseries \textcolor{red}{Proof:} }}

%symbol definitions
\newcommand{\un}[1]{\underline{#1}}
\newcommand{\mbZ}{\mathbb{Z}}
\newcommand{\mbR}{\mathbb{R}}
\newcommand{\mbN}{\mathbb{N}}
\newcommand{\mbQ}{\mathbb{Q}}
\newcommand{\mbC}{\mathbb{C}}
\newcommand{\mbJ}{\mathbb{J}}
\newcommand{\mbF}{\mathbb{F}}
\newcommand{\mcC}{\mathcal{C}}
\newcommand{\mcJ}{\mathcal{J}}
\newcommand{\mcS}{\mathcal{S}}
\newcommand{\mcP}{\mathcal{P}}
\newcommand{\mcR}{\mathcal{R}}
\newcommand{\hra}{\hookrightarrow}
\newcommand{\tra}{\twoheadrightarrow}
\newcommand{\lra}{\leftrightarrow}
\newcommand{\ep}{\epsilon}
\newcommand{\Ra}{\Rightarrow}
\newcommand{\mb}[1]{\mathbb{#1}}
\newcommand{\mc}[1]{\mathcal{#1}}
\newcommand{\bfs}[1]{{\bfseries #1}}
\newcommand{\bs}[1]{\boldsymbol{#1}}
%Operator definitions
\DeclareMathOperator{\Irr}{Irr}
\DeclareMathOperator{\triv}{triv}
\DeclareMathOperator{\cyc}{cyc}
\DeclareMathOperator{\lcm}{lcm}
\DeclareMathOperator{\expo}{x}
\DeclareMathOperator{\ord}{o}
\DeclareMathOperator{\imm}{im}
\DeclareMathOperator{\sgn}{sgn}
\DeclareMathOperator{\Sym}{Sym}
\DeclareMathOperator{\alt}{alt}
\DeclareMathOperator{\irr}{irr}
\DeclareMathOperator{\eqt}{Equiv}
\DeclareMathOperator{\pat}{Part}
%\DeclareMathOperator{\sgn}{sgn}
%\DeclareMathOperator{\Aut}{Aut}
\DeclareMathOperator{\Gl}{Gl}
\DeclareMathOperator{\M}{M}
\DeclareMathOperator{\Id}{Id}
\DeclareMathOperator{\fixx}{Fix}
\DeclareMathOperator{\suppp}{Supp}
\DeclareMathOperator{\gl}{Gl}
\DeclareMathOperator{\id}{Id}
\DeclareMathOperator{\Aut}{Aut}
\DeclareMathOperator{\Inn}{Inn}
\DeclareMathOperator{\orb}{orb}
\DeclareMathOperator{\ii}{I}
\DeclareMathOperator{\im}{im}
\DeclareMathOperator{\Fix}{Fix}
\DeclareMathOperator{\Co}{Co}
\DeclareMathOperator{\md}{md}
\DeclareMathOperator{\qt}{qt}
\DeclareMathOperator{\ExtendedGCD}{ExtendedGCD}
\DeclareMathOperator{\Mod}{Mod}
\DeclareMathOperator{\GCD}{GCD}
\newcommand{\nms}{\negmedspace}
\newcommand{\nts}{\negthinspace}

\newcommand{\itep}{\item {\bfseries Problem}\ }
\newcommand{\gen}[1]{\langle \nts#1 \nts\rangle}
\newcommand{\quot}[2]{#1/ #2}
\newcommand{\order}[1]{\left|<\nts #1 \nts s>\right|}

%These next two commands are for making answers. 
\newcommand{\beans}{\begin{description} \item[{ \bfseries \textcolor{red}{Answer}}]\ }
\newcommand{\eans }{\end{description}}
%\newcommand{\begin{comment}ex}{{ \bfseries \textcolor{red}{Answer}}}

%To turn the answer into problem sets use replace to replace \begin{comment} with \begin{comment} and \\end{comment}  by \end{comment}.
\newcommand{\lieb}[3][{{}}]{\frac{d^#1 #2}{d\,#3^#1}}
\DeclarePairedDelimiter\abs{\lvert}{\rvert}
\DeclarePairedDelimiter\norm{\lVert}{\rVert}

\title{\textbf{Math 351 - Problem Set 1}}
\author{Guy Matz}
\date{\today}

\begin{document} 

\maketitle
\newpage % Q1

\section{Finite, Countable, and Uncountable Sets}
\begin{enumerate}[series=p]
\item Prove the following sets are Countable (i.e., there is a 1-1 mapping onto the set $\mbJ=\{1,2,3,4,...\}$)
\benu
	\item A={-2, -4, -6, -8, ...}\\
	A function $f$ is a 1-1 mapping if $f(p) = f(q) \implies p = q$\\
	$f(x) = -2x$\\
	$f(p) = f(q)$, so $-2p = -2q$ and $p = q$.  Hence $f$ is 1-1.
	\item B={-1, -3, -5, -7,...}\\
	$f(x) = 2x +1$\\
	$-2p + 1 = -2q +1$, so $p = q$ and $f$ is 1-1.
	\item C={-1, -4, -9, -16, ...}\\
	$f(x) = -(x^2)$\\
	$-(p^2) = -(q^2)$, so $p^2 = q^2$ and $p = q$ so $f$ is 1-1.
	
	My gut is telling me this doesn't feel right . . .
	
\eenu

\newpage

\item Show that the following sets are equivalent to \\
$A = \{x \in \mcR; 0 < x < 1\}$
\benu
	\item $B = \{x \in \mbR; 0 < x < 10\}$
	
	Two sets are equivalent if they are 1-1 and onto, i.e. bijective.
	
	$f(x) = 10x$
	
	\benu
		\item 1-1\\
			$f(p) = f(q)$\\
			$10p = 10q$\\
			$p = q$, so $f$ is 1-1.
		\item onto\\
			Given $y \in B$, we can find  $x \in A$ such that $f(x) = y$.  $f(x) = 10x = y$, so $x = \frac{y}{10}$, which is in $A$.  Hence $f$ is onto and $A \sim B$
	\eenu
	\item $C = \{x \in \mbR; −4 < x < −1\}$\\
	$f(x) = 4 - 3x$\\
	\benu
		\item 1-1\\
		$f(p) = f(q)$, so $4-3p = 4 - 3q$, and $p = q$.  Hence $f$ is 1-1.
		
		\item onto\\
		Given a $y \in C$ we can find a $x \in A$ such that $f(x) = y$.\\
		$f(x) = 4 - 3x = y$, so $x = \frac{4-y}{3}$ which is in $A$, so $f$ is onto.  Hence $A \sim C$.
	\eenu
\eenu
\newpage

\item
	\benu
		\item $f: \mbR \to \mbR$ defined by $f(x) = x^2$.\\
			\benu
				\item Find $f^{-1}(16)$\\
		
					$f^{-1}(16) = \{-4,4\}$\\
							
				
				\item $f^{-1}(U)$, where	$U = [9,16]$.\\
		
				$f^{-1}(U) = [-4,-3] \cup [3,4]$\\
			\eenu
			
		\item $f: \mbR^2 \to \mbR$ defined by $f(x,y) = x^2 + y^2$.\\  
			\benu
				\item Find $f^{-1}(0)$\\
				
				$f^{-1}(0) = (0,0)$\\
				
				\item $f^{-1}(1)$\\
					
				$f^{-1}(1) = (cos(x),sin(y))$ where $x = y$\\
				
				
				\item $f^{-1}(U)$, where	$U =(0,4)$.\\
					
				$f^{-1}(U) = 1/2(cos(x),sin(y))$ where $0 \leq x,y \leq \pi/2$
		\eenu
	\eenu
\newpage

\item $f: \mbR \to \mbR$ defined by $f(x) = x^2$ and let $U = (-1,1)$.
	\benu
		\item Find $f^{-1}(U)$.\\
		
		$f^{-1}(U) = f^{-1}( (-1, 1))$\\
		$f^{-1}(-1) = i \notin \mbR$\\
		$f^{-1}(0) = 0 \in \mbR$\\
		$f^{-1}(1) = 1 \in \mbR$\\
		$f^{-1}(U) = [0,1) \subset \mbR$ ??\\
										
		\item Find $f(f^{-1}(U))$.  (Notice that $f(f^{-1}(U)) \subseteq U$, but $f(f^{-1}(U)) \neq U)$\\
		
		$f(f^{-1}(U)) = f([0,1)) = [0,1)$ ?
	\eenu
\newpage




\end{enumerate}

\section{Metric Spaces}
\begin{enumerate}
\item  Prove from the definition of a metric space that $\mbR^2$, $d$ is a metric space where
$$d(p,q) = \abs{p_1 - q_1} + \abs{p_2 - q_2}; p = (p_1, p_2), q = (q_1, q_2)$$

\benu
	\item Non-negativity\\
	
	$d(p,q) = \abs{p_1 - q_1} + \abs{p_2 - q_2}> 0$ when $p \neq q$.  This is true by properties of absolute values.
	\\
	\item Identity of Indiscernibles\\
	\\
	$d(p,p) = \abs{p_1 - p_1} + \abs{p_2 - p_2} = 0$
	\\
	\item Symmetry
	\begin{align*}
	d(p,q) &= \abs{p_1 - q_1} + \abs{p_2 - q_2} \\
	&= \sqrt{(p_1 - q_1)^2} + \sqrt{(p_2 - q_2)^2}\\
	&= \sqrt{p_1^2 -2p_1q_1 + q_1^2} + \sqrt{(p_2^2 -2p_2q_2 + q_2)^2}\\
	&= \sqrt{q_1^2 -2q_1p_1 + p_1^2} + \sqrt{(q_2^2 -2q_2p_2 + p_2)^2}\\
	&= \sqrt{(q_1 - p_1)^2} + \sqrt{(q_2 - p_2)^2}\\
	&= \abs{q_1 - p_1} + \abs{q_2 - p_2} \\
	&= d(q,p)
	\end{align*}
	
	\item Triangle Inequality\\
	\begin{align*}
	d(p,q) &\leq d(p,r) + d(r,q)\\
	\abs{p_1 - q_1} + \abs{p_2 - q_2} &\leq \abs{p_1 - r_1} + \abs{p_2 - r_2} + \abs{r_1 - q_1} + \abs{r_2 - q_2}\\
	\end{align*}
	
	We will show for all real numbers that
	$$\abs{x + y} \leq \abs{x} + \abs{y}$$		
	\begin{align*}
	\abs{x + y} &= (x + y)^2 \\
	&= x^2 +2xy +y^2 \\
	&= \abs{x}^2 +2xy + \abs{y}^2
	\end{align*}
	
	But for any real number, $w$, $w \leq \abs{w}$; therefore
	\begin{align*}
	\abs{x + y}^2 &= \abs{x}^2 +2xy + \abs{y}^2 \\
	&\leq \abs{x}^2 + 2\abs{x}\abs{y} + \abs{y}^2 \\
	&= (\abs{x} + \abs{y})^2
	\end{align*}
	
	Taking square roots, we get:
	$$\abs{x + y} \leq \abs{x} + \abs{y}$$
	Now we let $x = \abs{p_1 - r_1} + \abs{p_2 - r_2}$ and $y = \abs{r_1 - q_1} + \abs{r_2 - q_2}$, so
	$$\abs{\abs{p_1 - r_1} + \abs{p_2 - r_2} + \abs{r_1 - q_1} + \abs{r_2 - q_2}} \leq \abs{\abs{p_1 - r_1} + \abs{p_2 - r_2}} + \abs{\abs{r_1 - q_1} + \abs{r_2 - q_2}}$$
	\\
	Thus, $A = \mbN, d(p,q) = \abs{p_1 - q_1} + \abs{p_2 - q_2}$ is a metric space.\\

\eenu

\newpage

\item Let A={positive Integers} and B={all Integers}. Let d be defined by:
$$d(p,q) = \abs{p^2 - q^2}.$$
\benu
\item Prove that A,d is a metric space.
	\benu
		\item Non-negativity\\
		
		$d(p,q) = \abs{p^2 - q^2} > 0$ when $p \neq q$.  This is true by properties of absolute values.
		\\
		\item Identity of Indiscernibles\\
		\\
		$d(p,p) = \abs{p^2 - p^2} = 0$
		\\
		\item Symmetry\\
		\begin{align*}
		d(p,q) &= \abs{p^2 - q^2} \\
		       &= \abs{(p+q)(p-q)} \\
		       &= \abs{(q+p)(q-p)} \\
		       &= \abs{q^2 - p^2} \\
		       &= d(q,p)
		\end{align*}
		
		\item Triangle Inequality\\
		\begin{align*}
		d(p,q) &\leq d(p,r) + d(r,q)\\
		\abs{p^2 - q^2} &\leq \abs{p^2 - r^2} + \abs{r^2 - q^2}\\
		\end{align*}
		
		We will show for all real numbers that
		$$\abs{x + y} \leq \abs{x} + \abs{y}$$		
		\begin{align*}
		\abs{x + y} &= (x + y)^2 \\
		            &= x^2 +2xy +y^2 \\
		            &= \abs{x}^2 +2xy + \abs{y}^2
		\end{align*}
		
		But for any real number, $w$, $w \leq \abs{w}$; therefore
		\begin{align*}
		\abs{x + y}^2 &= \abs{x}^2 +2xy + \abs{y}^2 \\
					&\leq \abs{x}^2 + 2\abs{x}\abs{y} + \abs{y}^2 \\
					&= (\abs{x} + \abs{y})^2
		\end{align*}
		
		Taking square roots, we get:
		$$\abs{x + y} \leq \abs{x} + \abs{y}$$
		Now we let $x = p^2 - r^2$ and $y = r^2 - q^2$, so
		$$\abs{p^2 - q^2} \leq \abs{p^2 - r^2} + \abs{r^2 - q^2}$$
		\\
		Thus, $A = \mbN, d(p,q) = \abs{p^2 - q^2}$ is a metric space.\\
	\eenu
	
\item Prove that $B,d$ is not a metric space.\\

$B,d$ is not not a metric space since there exists negative numbers such that for $q = -p$, i.e. $p \neq q$, $d(p,-p) = 0$, which violates $d(p,q) > 0, p \neq q$


\eenu

\newpage

\item Prove that $d((x_1 , y_1 ), (x_2 , y_2 )) = \abs{x_2 - x_1}$ is not a metric on $\mbR^2$.
\\

For $p = (x_1, y_1), q = (x_1, y_2), d(p,q) = 0$.  This violates the requirement that $d(p,q) > 0$ if $p \neq q$.

\newpage


\end{enumerate}
\end{document}
