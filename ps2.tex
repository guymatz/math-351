\documentclass[12pt]{amsart}
\setlength{\parskip}{.1in}
\setlength{\parindent}{0cm}
%myalterations
\usepackage{amssymb}
\usepackage{amsmath}
\usepackage[usenames,dvipsnames,svgnames,table]{xcolor}
\usepackage[colorlinks=true,urlcolor=blue,pdfborder={0 0 .5}pdfnewwindow=true]{hyperref}
\usepackage{enumitem}
%\usepackage{amsthm}
\usepackage{graphicx}
\usepackage{verbatim}
\usepackage{tabularx}
%\usepackage{arydshln,leftidx,mathtools}
\usepackage{bm}
\usepackage{tikz}
\usepackage{tikz-cd}
\usepackage{hyperref}
\usepackage{bm}
\usepackage{mathtools}
\usepackage{stmaryrd}
\usepackage{booktabs}  % for tables

%\setlength{\dashlinedash}{.4pt}
%\setlength{\dashlinegap}{.8pt}
%\usepackage{amsthm}
\usepackage{verbatim}
%\usepackage{commath}
%My commands
%environment abbreviations
\newcommand{\benu}{\begin{enumerate}}
\newcommand{\eenu}{\end{enumerate}}
\newcommand{\bed}{\begin{description}}
\newcommand{\ed}{\end{description}}
\theoremstyle{definition}
\newtheorem{theorem}{Theorem}
\newtheorem{notation}[theorem]{Notation}
\newtheorem{exercise}[theorem]{Exercise}
\newcommand{\bex}{\begin{exercise}}
\newcommand{\ex}{\end{exercise}}

\newcommand{\pru}{{ \bfseries \textcolor{red}{Proof:} }}

%symbol definitions
\newcommand{\un}[1]{\underline{#1}}
\newcommand{\mbZ}{\mathbb{Z}}
\newcommand{\mbR}{\mathbb{R}}
\newcommand{\mbN}{\mathbb{N}}
\newcommand{\mbQ}{\mathbb{Q}}
\newcommand{\mbC}{\mathbb{C}}
\newcommand{\mbF}{\mathbb{F}}
\newcommand{\mcC}{\mathcal{C}}
\newcommand{\mcS}{\mathcal{S}}
\newcommand{\mcP}{\mathcal{P}}
\newcommand{\mcR}{\mathcal{R}}
\newcommand{\hra}{\hookrightarrow}
\newcommand{\tra}{\twoheadrightarrow}
\newcommand{\lra}{\leftrightarrow}
\newcommand{\ep}{\epsilon}
\newcommand{\Ra}{\Rightarrow}
\newcommand{\mb}[1]{\mathbb{#1}}
\newcommand{\mc}[1]{\mathcal{#1}}
\newcommand{\bfs}[1]{{\bfseries #1}}
\newcommand{\bs}[1]{\boldsymbol{#1}}
%Operator definitions
\DeclareMathOperator{\Irr}{Irr}
\DeclareMathOperator{\triv}{triv}
\DeclareMathOperator{\cyc}{cyc}
\DeclareMathOperator{\lcm}{lcm}
\DeclareMathOperator{\expo}{x}
\DeclareMathOperator{\ord}{o}
\DeclareMathOperator{\imm}{im}
\DeclareMathOperator{\sgn}{sgn}
\DeclareMathOperator{\Sym}{Sym}
\DeclareMathOperator{\alt}{alt}
\DeclareMathOperator{\irr}{irr}
\DeclareMathOperator{\eqt}{Equiv}
\DeclareMathOperator{\pat}{Part}
%\DeclareMathOperator{\sgn}{sgn}
%\DeclareMathOperator{\Aut}{Aut}
\DeclareMathOperator{\Gl}{Gl}
\DeclareMathOperator{\M}{M}
\DeclareMathOperator{\Id}{Id}
\DeclareMathOperator{\fixx}{Fix}
\DeclareMathOperator{\suppp}{Supp}
\DeclareMathOperator{\gl}{Gl}
\DeclareMathOperator{\id}{Id}
\DeclareMathOperator{\Aut}{Aut}
\DeclareMathOperator{\Inn}{Inn}
\DeclareMathOperator{\orb}{orb}
\DeclareMathOperator{\ii}{I}
\DeclareMathOperator{\im}{im}
\DeclareMathOperator{\Fix}{Fix}
\DeclareMathOperator{\Co}{Co}
\DeclareMathOperator{\md}{md}
\DeclareMathOperator{\qt}{qt}
\DeclareMathOperator{\ExtendedGCD}{ExtendedGCD}
\DeclareMathOperator{\Mod}{Mod}
\DeclareMathOperator{\GCD}{GCD}
\newcommand{\nms}{\negmedspace}
\newcommand{\nts}{\negthinspace}

\newcommand{\itep}{\item {\bfseries Problem}\ }
\newcommand{\gen}[1]{\langle \nts#1 \nts\rangle}
\newcommand{\quot}[2]{#1/ #2}
\newcommand{\order}[1]{\left|<\nts #1 \nts s>\right|}

%These next two commands are for making answers. 
\newcommand{\beans}{\begin{description} \item[{ \bfseries \textcolor{red}{Answer}}]\ }
\newcommand{\eans }{\end{description}}
%\newcommand{\begin{comment}ex}{{ \bfseries \textcolor{red}{Answer}}}

%To turn the answer into problem sets use replace to replace \begin{comment} with \begin{comment} and \\end{comment}  by \end{comment}.
\newcommand{\lieb}[3][{{}}]{\frac{d^#1 #2}{d\,#3^#1}}
\DeclarePairedDelimiter\abs{\lvert}{\rvert}
\DeclarePairedDelimiter\norm{\lVert}{\rVert}

\title{\textbf{Math 351 - Problem Set 2}}
\author{Guy Matz}
\date{\today}

\begin{document} 

\maketitle
%\newpage % Q1

\section{Metric Spaces}
\begin{enumerate}[series=p]
\setcounter{enumi}{3}
\item
	\benu
		\item Define a metric on $\mbR$ by $d(p, q) = \abs{e^p - e^q}$. Find all of the points $p \in \mbR$ such that $d(p, 2) < 1$.
		\item Using the metric in problem number 1, find the set of all points
		$p = (p 1 , p 2 ) \in \mbR^2$ such that $d(p, O) \leq 1$, where $O = (0, 0)$. Sketch this set in $\mbR^2$ .
	\eenu
\newpage


\item Let $X = C[0,1]=$ the set of real valued, continuous functions on [0,1].  Define 2 metrics on $X$ by $d_1(f(x), g(x)) = \int_{0}^{1} \abs{f (x) - g(x)}dx$ and
$d_2(f(x), g(x)) =$ $\max\limits_{x \in [0,1]} |f(x) - g(x)|$. Let $f(x) = x$ and $g(x) = x^2$.
Find $d_1(f(x), g(x))$, and $d_2(f(x), g(x))$.

\newpage

\item Make a table of the subsets of $\mbR^2$ below (using the standard metric on $\mbR^2$ ) where across the top you have the categories: “closed”, “open”, “isolated points”, “Limit Points”, and “bounded”, and along the left side you have the sets A through H. Identify all limit points and isolated points. Put “Y” or “N” for the rest.
	\benu
		\item $A = {(x, y)| x^2 + y^2 \leq 1}$
		\item $B = {(x, y)| 0 < x^2 + y^2 \leq 1}$
		\item $C = {(x, y)| 0 < x^2 + y^2 < 1}$
		\item $D = {(x, y)|	\frac{1}{2}	< x^2 + y^2 < 1} \cup {(0,0)}$
		\item $E = {(x, y)| 0 < x < 2, y = 1}$
		\item $F = {(x, y)| 0 \leq x \leq 2, 0 \leq y \leq 1}$
		\item $G = {(x, y)| 1 \leq x, 1 \leq y \leq 2}$
		\item $H = {(0,0), (0,1), (1,0)}$
	\eenu
	
	\begin{table}[h!]
		\centering
		\caption{Caption for the table.}
		\label{tab:table1}
		\begin{tabular}{c|c|c|c|c|c}
			Subset & Closed & Open & Iso. pts. & Limit pts. & Bounded\\
			\toprule
			    A & &  &  & &\\ \hline
			    B & &  &  & &\\ \hline
			    C & &  &  & &\\ \hline
			    D & &  &  & &\\ \hline
			    E & &  &  & &\\ \hline
			    F & &  &  & &\\ \hline
			    G & &  &  & &\\ \hline
			    H & &  &  & &\\
			
		\end{tabular}
	\end{table}

\newpage

\item Prove the following (assume the standard metric on $\mbR$ and $\mbR^2$):
	\benu
		\item $(-2,2)$ is an open set in $\mbR$.
		\item $[-2,2]$ is a closed set in $\mbR$.
		\item $(-2,2]$ is neither an open set nor a closed set in $\mbR$.
		\item Is $A = \{(x, y)| -2 < x < 2, y = 0\}$ open in $\mbR^2$? Prove your answer.
	\eenu

\newpage

\item Let $A,B, C \subset X$, $d$ be non-empty open sets in a metric space $X$. Prove the
following (without using the theorem that states that the union of open sets is open and the finite intersection of open sets is open).
	\benu
		\item $A \cup B \cup C$ is open in $X$
		\item $A \cap B \cap C$ is open in $X$
	\eenu
\newpage


\item Prove that If $X, d$ is a metric space and $E \subset F \subset X$, then $\overline{E}̅ \subset \overline{F}$.


\newpage

\end{enumerate}
\end{document}
